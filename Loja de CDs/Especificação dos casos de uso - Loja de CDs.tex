%% abtex2-modelo-glossarios.tex, v-1.9.6 laurocesar
%% Copyright 2012-2016 by abnTeX2 group at http://www.abntex.net.br/
%%
%% This work may be distributed and/or modified under the
%% conditions of the LaTeX Project Public License, either version 1.3
%% of this license or (at your option) any later version.
%% The latest version of this license is in
%%   http://www.latex-project.org/lppl.txt
%% and version 1.3 or later is part of all distributions of LaTeX
%% version 2005/12/01 or later.
%%
%% This work has the LPPL maintenance status `maintained'.
%%
%% The Current Maintainer of this work is the abnTeX2 team, led
%% by Lauro César Araujo. Further information are available on
%% http://www.abntex.net.br/
%%
%% This work consists of the files abtex2-modelo-glossarios.tex,
%% abntex2-modelo-include-comandos and abntex2-modelo-references.bib
%%
                     
% ------------------------------------------------------------------------
% ------------------------------------------------------------------------
% abnTeX2: Exemplo de glossários com o pacote glossaries e abntex2
% ------------------------------------------------------------------------
% ------------------------------------------------------------------------
 
\documentclass[
	% -- opções da classe memoir --
	12pt,				% tamanho da fonte
	openright,			% capítulos começam em pág ímpar (insere página vazia caso preciso)
	twoside,			% para impressão em recto e verso. Oposto a oneside
	a4paper,			% tamanho do papel. 
	% -- opções da classe abntex2 --
	%chapter=TITLE,		% títulos de capítulos convertidos em letras maiúsculas
	%section=TITLE,		% títulos de seções convertidos em letras maiúsculas
	%subsection=TITLE,	% títulos de subseções convertidos em letras maiúsculas
	%subsubsection=TITLE,% títulos de subsubseções convertidos em letras maiúsculas
	% -- opções do pacote babel --
	english,			% idioma adicional para hifenização
	french,				% idioma adicional para hifenização
	spanish,			% idioma adicional para hifenização
	brazil,				% o último idioma é o principal do documento
	]{abntex2}
 
 
% ---
% PACOTES
% ---
 
% ---
% Pacotes fundamentais
% ---
\usepackage{lmodern}			% Usa a fonte Latin Modern			
\usepackage[T1]{fontenc}		% Selecao de codigos de fonte.
\usepackage[utf8]{inputenc}		% Codificacao do documento (conversão automática dos acentos)
\usepackage{indentfirst}		% Indenta o primeiro parágrafo de cada seção.
\usepackage{color}				% Controle das cores
\usepackage{graphicx}			% Inclusão de gráficos
\usepackage{microtype} 			% para melhorias de justificação
% ---

% ---
% Pacotes glossaries
% ---
%\usepackage[subentrycounter,seeautonumberlist,nonumberlist=true]{glossaries}
% para usar o xindy ao invés do makeindex:
%\usepackage[xindy={language=portuguese},subentrycounter,seeautonumberlist,nonumberlist=true]{glossaries}
% ---

% ---
% Pacotes de citações
% ---
\usepackage[brazilian,hyperpageref]{backref}	 % Paginas com as citações na bibl
\usepackage[alf]{abntex2cite}	% Citações padrão ABNT
 
% ---
% Informações de dados para CAPA e FOLHA DE ROSTO
% ---
\titulo{Detalhamento dos Requisitos da Loja de CDs}
\autor{César A. Magalhães}
\local{Brasil}
\data{2016}
%\orientador{Lauro César Araujo}
%\coorientador{Equipe \abnTeX}
\instituicao{%
  Universidade Norte do Paraná -- UNOPAR}
%\tipotrabalho{Especificação de casos de uso}
% O preambulo deve conter o tipo do trabalho, o objetivo,
% o nome da instituição e a área de concentração
\preambulo{Especificação de Caso de Uso.}
% ---
 
 
% ---
% Configurações de aparência do PDF final
 
% alterando o aspecto da cor azul
\definecolor{blue}{RGB}{41,5,195}
 
% informações do PDF
\makeatletter
\hypersetup{
     	%pagebackref=true,
		pdftitle={\@title}, 
		pdfauthor={\@author},
    	pdfsubject={\imprimirpreambulo},
	    pdfcreator={LaTeX with abnTeX2},
		pdfkeywords={abnt}{latex}{abntex}{abntex2}{glossários}, 
		colorlinks=true,       		% false: boxed links; true: colored links
    	linkcolor=blue,          	% color of internal links
    	citecolor=blue,        		% color of links to bibliography
    	filecolor=magenta,      		% color of file links
		urlcolor=blue,
		bookmarksdepth=4
}
\makeatother

% ---
 
% ---
% Espaçamentos entre linhas e parágrafos
% ---
 
% O tamanho do parágrafo é dado por:
\setlength{\parindent}{1.3cm}
 
% Controle do espaçamento entre um parágrafo e outro:
\setlength{\parskip}{0.2cm}  % tente também \onelineskip
 
% ---
% compila o indice
% ---
\makeindex
% ---
 
% ---
% GLOSSARIO
% ---
%\makeglossaries
 
% ---
% entradas do glossario
% ---
% \newglossaryentry{pai}{
%                name={pai},
%                plural={pai},
%                description={este é uma entrada pai, que possui outras
%                subentradas.} }

% \newglossaryentry{componente}{
%                name={componente},
%                plural={componentes},
%                parent=pai,
%                description={descriação da entrada componente.} }
 
% \newglossaryentry{filho}{
%                name={filho},
%                plural={filhos},
%                parent=pai,
%                description={isto é uma entrada filha da entrada de nome
%                \gls{pai}. Trata-se de uma entrada irmã da entrada
%                \gls{componente}.} }
 
%\newglossaryentry{equilibrio}{
%                name={equilíbrio da configuração},
%                see=[veja também]{componente},
%                description={consistência entre os \glspl{componente}}
%                }

%\newglossaryentry{latex}{
%                name={LaTeX},
%                description={ferramenta de computador para autoria de
%                documentos criada por D. E. Knuth} }

%\newglossaryentry{abntex2}{
%                name={abnTeX2},
%                see=latex,
%                description={suíte para LaTeX que atende os requisitos das
%                normas da ABNT para elaboração de documentos técnicos e científicos brasileiros} }
% ---

% ---
% Exemplo de configurações do glossairo
%\renewcommand*{\glsseeformat}[3][\seename]{\textit{#1}  
% \glsseelist{#2}}
% ---
              
                
% ----
% Início do documento
% ----
\begin{document}
 
% Retira espaço extra obsoleto entre as frases.
\frenchspacing
 
% ----------------------------------------------------------
% ELEMENTOS PRÉ-TEXTUAIS
% ----------------------------------------------------------
 
% ---
% Capa
% ---
\imprimircapa
% ---
 
% ---
% inserir o sumario
% ---
\pdfbookmark[0]{\contentsname}{toc}
\tableofcontents*
\cleardoublepage
% ---
  
 
% ----------------------------------------------------------
% ELEMENTOS TEXTUAIS
% ----------------------------------------------------------
\textual
 
% ----------------------------------------------------------
% Introdução
% ----------------------------------------------------------
\chapter*[Introdução]{Introdução}
\addcontentsline{toc}{chapter}{Introdução}
 
Uma loja de CDs possui discos para venda. Um cliente pode comprar uma quantidade ilimitada de discos, mas para isto ele deve se dirigir à loja. Essa possui um atendente cuja função é atender os clientes durante a venda dos discos. Ela também possui um gerente cuja função é administrar o estoque para que não faltem produtos. Além disso, esse é quem dá folga ao atendente, ou seja, o gerente também atende os cliente durante a venda dos discos.

A venda pode ser à vista ou a prazo. Em ambos os casos o estoque é atualizado e uma nota fiscal é entregue ao consumidor.
\begin{itemize}
	\item No caso de uma venda à vista, cliente cadastrado que compra mais de 5 CDs de uma só vez ganha um desconto de 1\% para cada ano de cadastro na loja.
	\item No caso de uma venda a prazo, essa pode ser parcelada em 2 pagamentos com um acréscimo de 20\%. A venda a prazo pode ser paga com cartão ou boleto. Para pagamento com boleto bancário, o mesmo é gerado e entregue ao cliente, armazenando-os no sistema para lançamento posterior no caixa. Para pagamento com cartão, o cliente com mais de 10 anos de cadastro na loja ganha o mesmo desconto da compra à vista.
\end{itemize}

\chapter{Vender CDs - UC001} \label{uc001}
 
\section{Breve descrição}

Atendente vende um ou mais CDs à um usuário.

\section{Atores}

\begin{enumerate}
	\item Atendente.
\end{enumerate}

\section{Pré-condições}

\begin{enumerate}
	\item O Atendente deve estar logado no sistema.
	\item O Atendente deve ter acesso à função.
\end{enumerate}

\section{Fluxo de evento}

\subsection{Fluxo básico}

\begin{enumerate}	
	\item O Atendente seleciona a opção \textbf{Vender CDs}.
	\item O Sistema exibe a lista de CDs. \label{uc001_fluxo_basico:2}
	\item O Atendente seleciona os CDs, informando as respectivas quantidades.
	\item O Sistema exibe a lista de clientes. \label{uc001_fluxo_basico:4}
	\item O Atendente seleciona o cliente.
	\item O Atendente seleciona a opção \textbf{Vender}.
	\item O Sistema exibe as informações da venda: CDs, quantidades e o cliente.
	\item O Atendente confirma as informações da venda. \label{uc001_fluxo_basico:8}
	\item O Sistema efetua a venda, verificando a regra RN\ref{uc001_rn:1}. \label{uc001_fluxo_basico:9}
	\begin{enumerate}
		\item O Atendente seleciona o tipo de venda: \textbf{A prazo} ou \textbf{À Vista}.
		\item O Sistema deve executar o caso de uso \nameref{uc002} ou o caso de uso \nameref{uc003}, de acordo com opção selecionada pelo Atendente no passo anterior.
		\item O sistema atualiza o estoque de acordo com a regra RN\ref{uc001_rn:2}. 
	\end{enumerate}
	\item O sistema emite a Nota fiscal conforme à ED\ref{uc001_ed:1}.
	\item O caso de uso é encerrado.  
\end{enumerate}

\subsection{Fluxo alternativo}

\begin{enumerate}
	\item Alternativa ao passo \ref{uc001_fluxo_basico:4} - Cliente não cadastrado:
	\begin{enumerate}
		\item O Atendente seleciona a opção \textbf{Cadastrar Cliente}.
		\item O Sistema executa o caso de uso \nameref{uc008}.
		\item O Sistema retorna ao passo \ref{uc001_fluxo_basico:4}.
	\end{enumerate}
	\item Alternativa ao passo \ref{uc001_fluxo_basico:8} - Informações incorretas:
	\begin{enumerate}
		\item O Atendente não confirma as informações de venda.
		\item O Sistema retorna ao passo \ref{uc001_fluxo_basico:2}.
	\end{enumerate}
	\item Alternativa ao passo \ref{uc001_fluxo_basico:9} - A regra RN\ref{uc001_rn:1} não é atendida:
	\begin{enumerate}
		\item O Sistema exibe a mensagem: \textbf{Não há produtos disponíveis em estoque}.
		\item O caso de uso é encerrado.
	\end{enumerate}
\end{enumerate}


\section{Pós-condições}

\begin{enumerate}
	\item Venda efetuada, se as regras de validação forem verificadas.
\end{enumerate}

\section{Regras de negócio}
\begin{enumerate}
	\item O produto deve está disponivel em estoque. \label{uc001_rn:1}
	\item O sistema deve atualizar o estoque de produtos: para cada produto selecionado para venda, o sistema deve subtrair a quantidade disponível em estoque. \label{uc001_rn:2}
\end{enumerate}

\section{Estrutura de dados}

\begin{enumerate}
	\item Nota fiscal: \label{uc001_ed:1}
	\begin{enumerate}
		\item CPF do cliente.
		\item Nome do cliente.
		\item Endereço do cliente.
		\item CNPJ da loja.
		\item Razão social da loja.
		\item Endereço da loja.
		\item Data da compra.
		\item Código dos produtos comprados.
		\item Descrição dos produtos comprados.
		\item Valores dos produtos comprados.
		\item Valor total da compra.
		\item Valor do desconto.
		\item Valor final da compra.
	\end{enumerate}
\end{enumerate}



%\section{Requerimentos especiais}

%Não aplicável.

%\section{Pontos de extensão}

%Nenhum.
\chapter{Vender CDs a prazo - UC002} \label{uc002}
\chapter{Vender CDs à vista - UC003} \label{uc003}
\chapter{Cadastrar cliente - UC008} \label{uc008} 
 
% ----------------------------------------------------------
% ELEMENTOS PÓS-TEXTUAIS
% ----------------------------------------------------------
\postextual
 
 
% ----------------------------------------------------------
% Referências bibliográficas
% ----------------------------------------------------------
%\bibliography{referencias}
 
% ----------------------------------------------------------
% Glossário
% ----------------------------------------------------------

% ---
% Define nome e preâmbulo do glossário
% ---
%\phantompart
%\renewcommand{\glossaryname}{Glossário}
%\renewcommand{\glossarypreamble}{Esta é a descrição do glossário. Experimente visualizar outros estilos de glossários, como o \texttt{altlisthypergroup}, por exemplo.\\\\}

% ---
% Traduções para o ambiente glossaries
% ---
%\providetranslation{Glossary}{Glossário}
%\providetranslation{Acronyms}{Siglas}
%\providetranslation{Notation (glossaries)}{Notação}
%\providetranslation{Description (glossaries)}{Descrição}
%\providetranslation{Symbol (glossaries)}{Símbolo}
%\providetranslation{Page List (glossaries)}{Lista de Páginas}
%\providetranslation{Symbols (glossaries)}{Símbolos}
%\providetranslation{Numbers (glossaries)}{Números} 
% ---

% ---
% Estilo de glossário
% ---
% \setglossarystyle{index}
% \setglossarystyle{altlisthypergroup}
% \setglossarystyle{tree}


% ---
% Imprime o glossário
% ---
%\cleardoublepage
%\phantomsection
%\addcontentsline{toc}{chapter}{\glossaryname}
%\printglossaries
% ---
  
 
% ----------------------------------------------------------
% Apêndices
% ----------------------------------------------------------
 
% ----------------------------------------------------------
% Anexos
% ----------------------------------------------------------
 
%---------------------------------------------------------------------
% INDICE REMISSIVO
%---------------------------------------------------------------------
 
\end{document}
